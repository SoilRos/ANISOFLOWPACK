\chapter{Introduction}

ANISOFLOW is a program that solves the groundwater flow in a computing parallel environment. The groundwater equation is solved with three different finite difference methods, where is taken into account heterogeneity and anisotropy  mediums.\\

The main philosophy of ANISOFLOW is to capture the anisotropies of the medium into the flow, for that purpose the program has three ways to carry the calculus. The first one is the most usual method used to solve the groundwater equation, a scheme of seven blocks on finite difference, used by MODFLOW and the most popular programs to analyze aquifers. The second one is the finite difference equation proposed by Li, et al (2014) using a scheme of 19 blocks. And the last one is a finite difference scheme of 19 blocks we have proposed to attack the anisotropy of the mediums. \\

ANISOFLOW was built over PETSc libraries, which means ANISOFLOW has the PETSc advantages on the resolution of the linear systems: grid management, methods to solve the systems, preconditioning techniques to accelerate iterative processes, monitoring of solving process, portability, among others.\\

