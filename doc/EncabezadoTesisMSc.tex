\documentclass[12pt,spanish,fleqn,openany,letterpaper,pagesize,final]{scrbook}
%\documentclass[12pt,spanish,fleqn,openany,letterpaper,pagesize,draft]{scrbook}


%\usepackage[ansinew]{inputenc}%	<--	old
\usepackage[spanish]{babel}
\usepackage[utf8]{inputenc}%		<-- new
\usepackage{color}%					<--	new
\usepackage{enumitem}%				<--	new
\usepackage{gensymb}%				<-- new
\usepackage[natbibapa]{apacite}%	<-- new
\usepackage{fancyhdr}
\usepackage{epsfig}
\usepackage{epic}
\usepackage{eepic}
\usepackage{amsmath,amssymb}
\usepackage{threeparttable}
\usepackage{amscd}
\usepackage{here}
\usepackage{graphicx}
\usepackage{lscape}
\usepackage{tabularx}
\usepackage{subfigure}
\usepackage{longtable}
\usepackage[T1]{fontenc}%			<--	new
\usepackage{listings}%				<--	new
\usepackage{algorithm}%				<--	new
\usepackage{algorithmic}%			<--	new
\floatname{algorithm}{Algoritmo}
\renewcommand{\listalgorithmname}{Lista de algoritmos}
\renewcommand{\algorithmicrequire}{\textbf{Entrada:}}
\renewcommand{\algorithmicensure}{\textbf{Salida:}}
\renewcommand{\algorithmicend}{\textbf{fin}}
\renewcommand{\algorithmicif}{\textbf{si}}
\renewcommand{\algorithmicthen}{\textbf{entonces}}
\renewcommand{\algorithmicelse}{\textbf{si no}}
\renewcommand{\algorithmicelsif}{\algorithmicelse,\ \algorithmicif}
\renewcommand{\algorithmicendif}{\algorithmicend\ \algorithmicif}
\renewcommand{\algorithmicfor}{\textbf{para}}
\renewcommand{\algorithmicforall}{\textbf{para todo}}
\renewcommand{\algorithmicdo}{\textbf{hacer}}
\renewcommand{\algorithmicendfor}{\algorithmicend\ \algorithmicfor}
\renewcommand{\algorithmicwhile}{\textbf{mientras}}
\renewcommand{\algorithmicendwhile}{\algorithmicend\ \algorithmicwhile}
\renewcommand{\algorithmicloop}{\textbf{repetir}}
\renewcommand{\algorithmicendloop}{\algorithmicend\ \algorithmicloop}
\renewcommand{\algorithmicrepeat}{\textbf{repetir}}
\renewcommand{\algorithmicuntil}{\textbf{hasta que}}
\renewcommand{\algorithmicprint}{\textbf{imprimir}} 
\renewcommand{\algorithmicreturn}{\textbf{devolver}} 
\renewcommand{\algorithmictrue}{\textbf{cierto }} 
\renewcommand{\algorithmicfalse}{\textbf{falso }} 
 % mi archivo de traducción
%\usepackage{draftwatermark}%		<--	new
%\SetWatermarkText{BIO-HAZARD}%		<--	new


\usepackage{rotating} %Para rotar texto, objetos y tablas seite. No se ve en DVI solo en PS. Seite 328 Hundebuch
                        %se usa junto con \rotate, \sidewidestable ....


\renewcommand{\theequation}{\thechapter-\arabic{equation}}
\renewcommand{\thefigure}{\textbf{\thechapter-\arabic{figure}}}
\renewcommand{\thetable}{\textbf{\thechapter-\arabic{table}}}


\pagestyle{fancyplain}%\addtolength{\headwidth}{\marginparwidth}
\textheight22.5cm \topmargin0cm \textwidth16.5cm
\oddsidemargin0.5cm \evensidemargin-0.5cm%
\renewcommand{\chaptermark}[1]{\markboth{\thechapter\; #1}{}}
\renewcommand{\sectionmark}[1]{\markright{\thesection\; #1}}
\lhead[\fancyplain{}{\thepage}]{\fancyplain{}{\rightmark}}
\rhead[\fancyplain{}{\leftmark}]{\fancyplain{}{\thepage}}
\fancyfoot{}
\thispagestyle{fancy}%


\addtolength{\headwidth}{0cm}
\unitlength1mm %Define la unidad LE para Figuras
\mathindent0cm %Define la distancia de las formulas al texto,  fleqn las descentra
\marginparwidth0cm
\parindent0cm %Define la distancia de la primera linea de un parrafo a la margen

%Para tablas,  redefine el backschlash en tablas donde se define la posici\'{o}n del texto en las
%casillas (con \centering \raggedright o \raggedleft)
\newcommand{\PreserveBackslash}[1]{\let\temp=\\#1\let\\=\temp}
\let\PBS=\PreserveBackslash

%Espacio entre lineas
\renewcommand{\baselinestretch}{1.1}

%Neuer Befehl f\"{u}r die Tabelle Eigenschaften der Aktivkohlen
\newcommand{\arr}[1]{\raisebox{1.5ex}[0cm][0cm]{#1}}

%Neue Kommandos
\usepackage{Befehle}


%Trennungsliste
\hyphenation {Reaktor-ab-me-ssun-gen Gas-zu-sa-mmen-set-zung
Raum-gesch-win-dig-keit Durch-fluss Stick-stoff-gemisch
Ad-sorp-tions-tem-pe-ra-tur Klein-schmidt
Kohlen-stoff-Mole-kular-siebe Py-rolysat-aus-beu-te
Trans-port-vor-gan-ge}


%%%% <-- new

\makeatletter
\newsavebox\myboxA
\newsavebox\myboxB
\newlength\mylenA

\newcommand*\xoverline[2][0.75]{%
    \sbox{\myboxA}{$\m@th#2$}%
    \setbox\myboxB\null% Phantom box
    \ht\myboxB=\ht\myboxA%
    \dp\myboxB=\dp\myboxA%
    \wd\myboxB=#1\wd\myboxA% Scale phantom
    \sbox\myboxB{$\m@th\overline{\copy\myboxB}$}%  Overlined phantom
    \setlength\mylenA{\the\wd\myboxA}%   calc width diff
    \addtolength\mylenA{-\the\wd\myboxB}%
    \ifdim\wd\myboxB<\wd\myboxA%
       \rlap{\hskip 0.5\mylenA\usebox\myboxB}{\usebox\myboxA}%
    \else
        \hskip -0.5\mylenA\rlap{\usebox\myboxA}{\hskip 0.5\mylenA\usebox\myboxB}%
    \fi}
\makeatother

%%%%

%%%% <-- new
% Para las citas al inicio del capitulo
% customize dictum format:
\setkomafont{dictumtext}{\itshape\small}
\setkomafont{dictumauthor}{\normalfont}
\renewcommand*\dictumwidth{\linewidth}
\renewcommand*\dictumauthorformat[1]{--- #1}
\renewcommand*\dictumrule{}
%%%%

\usepackage[margin=1in]{geometry} %Paquete para hacer nuevos entornos teo,def,etc



%%
\newenvironment{teorema}[2][Teorema]{\begin{trivlist}
\item[\hskip \labelsep {\scshape \bfseries #1}\hskip \labelsep {\scshape #2.}]}{\end{trivlist}}

\newenvironment{definicion}[2][Definición]{\begin{trivlist}
\item[\hskip \labelsep { \scshape \bfseries #1}\hskip \labelsep {\scshape #2.}]}{\end{trivlist}}

\newenvironment{lemma}[2][Lemma]{\begin{trivlist}
\item[\hskip \labelsep {\scshape \bfseries #1}\hskip \labelsep {\scshape #2.}]}{\end{trivlist}}

\newenvironment{reflection}[2][Reflection]{\begin{trivlist}
\item[\hskip \labelsep {\scshape \bfseries #1}\hskip \labelsep {\scshape #2.}]}{\end{trivlist}}

\newenvironment{proposicion}[2][Proposición]{\begin{trivlist}
\item[\hskip \labelsep {\scshape \bfseries #1}\hskip \labelsep {\scshape #2.}]}{\end{trivlist}}

\newenvironment{corollary}[2][Corollary]{\begin{trivlist}
\item[\hskip \labelsep {\scshape \bfseries #1}\hskip \labelsep {\scshape #2.}]}{\end{trivlist}}

\newenvironment{problema}[2][Problema]{\begin{trivlist}
\item[\hskip \labelsep {\scshape \bfseries #1}\hskip \labelsep {\scshape #2.}]}{\end{trivlist}}
%%%

%% Simbolos de finalizacion de demostracion y teorema
\newcommand*{\QEDA}{\hfill\ensuremath{\blacksquare}}% proofs
\newcommand*{\QEDB}{\hfill\ensuremath{\square}}% Defs y Teor
%%